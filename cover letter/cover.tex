\documentclass[11pt]{article}
\usepackage[utf8]{inputenc}
\usepackage{vmargin}
\usepackage{graphicx}
\usepackage{parskip}
\usepackage{url}
\usepackage{hyperref}

\setmarginsrb{2.25cm}{0.5cm}{2.25cm}{0.5cm}
{12pt}{20pt}{12pt}{36pt}

%----------------------------------------------------
% Sender info
%----------------------------------------------------
\def\Who{Dr. Kas{\i}m Sinan Y{\i}ld{\i}r{\i}m} 
\def\What{}
\def\Where{Embedded Software Group}
\def\Affiliation{Delft University of Technology}
\def\Address{Mekelweg 4, 2628 CD, Delft, The Netherlands}
\def\Email{E-mail:  k.s.yildirim@tudelft.nl}
\def\Telephone{Phone: +31 15 278 9375}
\def\Webpage{\url{http://www.st.ewi.tudelft.nl/~sinan}}

\def\Sender{
	
	\begin{flushright}
		\Who \What \\
		\Where \\
		\Affiliation \\
		\Address \\
		\Email \\
		\Telephone \\
		\Webpage
	\end{flushright}	
}

\begin{document}
	
	\thispagestyle{empty}
	
	\Sender
	
	\bigskip
	
	
	Dear Editor,
	
	With this letter we are submitting an article titled \emph{On the Synchronization of Computational RFIDs}.
	The objective of this article is to provide initial observations pertaining to 
	the synchronization of battery-free computational RFID platforms. In 
	particular, we design and implement the first synchronization protocol for the 
	WISP (Wireless 
	Identification and Sensing Platform) and provide real-world experiments, as 
	well as mathematical analysis to evaluate its performance. We also 
	provide an overview of new 
	research opportunities in this emerging domain.
	
	
	
	A preliminary version of this article was recently presented at the \emph{HLPC 
		2016--Hilariously 
		Low-Power Computing Pushing the Boundaries of Intermittently Powered Devices 
		Workshop}, 
	in Atlanta, GA, USA, 2 April, 2016; which is available online at 
	\href{https://arxiv.org/abs/1606.01719}{https://arxiv.org/abs/1606.01719} and 
	also attached to this submission. This submission extends the above mentioned 
	paper by providing 
	\begin{itemize}
		\item a more detailed related work in Section 2;
		\item extended descriptions of the hardware platform and testbed in Section 
		3;
		\item a mathametical framework in Section 4 to analyze the proposed 
		synchronization algorithms;
		\item extended descriptions of the proposed synchronization algorithms in 
		Section 5 and 6;
		\item a rigorous theoretical analysis of the proof of convergence and 
		steady-state error performance of the proposed algorithm in Appendix;
		\item an experimental work to evaluate the synchronization
		performance of the proposed solutions under the loss of computational 
		state in Section 7-3;	
		\item an experimental work to evaluate the energy overhead of the proposed solutions in Section 7-4;	
		\item implementation and evaluation of the synchronization of 
		many WISP tags in Section 8.
	\end{itemize}
	
	We believe that \emph{IEEE Transactions on Mobile Computing} is 
	the best 
	venue to present our work, as its scope 
	perfectly matches with the scope of our contribution. Thank you in advance for 
	considering our submission.
	
	\vspace{0.3cm}
	
	Sincerely,
	
	\Who \\
	Corresponding Author
	
	encl: \\
	- Article for review \\
	- The preliminary version of our article which has been presented at HLPC 	
	2016 Workshop\\  
	
\end{document}
