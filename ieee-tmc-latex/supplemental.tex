\documentclass[10pt,journal,compsoc]{IEEEtran}

\ifCLASSOPTIONcompsoc
  % IEEE Computer Society needs nocompress option
  % requires cite.sty v4.0 or later (November 2003)
  \usepackage[nocompress]{cite}
\else
  % normal IEEE
  \usepackage{cite}
\fi

\usepackage{array}
\usepackage{multirow}
\usepackage{amsmath}
\usepackage{amssymb}
\usepackage{stmaryrd}

\newcommand{\expected}[1]{E\!\left\llbracket #1 \right\rrbracket}

\addtocounter{equation}{7}


\begin{document}

\title{On the Synchronization of Computational RFIDs \\
(Supplemental Material)}

\author{Kas{\i}m~Sinan~Y{\i}ld{\i}r{\i}m, 
	Henko~Aantjes, 
	Przemys{\l}a~Pawe{\l}czak~\IEEEmembership{Member,~IEEE,} 
	and~Amjad~Yousef~Majid
	%
	\IEEEcompsocitemizethanks{
		%\IEEEcompsocthanksitem A preliminary version of this article was presented at the HLPC 2016 Workshop, Atlanta, GA, USA, April 2, 2016~\cite{hlpc2016}. \protect\\
	\IEEEcompsocthanksitem Kas{\i}m Sinan Y{\i}ld{\i}r{\i}m, Przemys{\l}aw Pawe{\l}czak and Amjad Yousef Majid are with the Embedded Software Group, Delft University of Technology, Mekelweg 4, 2628 CD Delft, The Netherlands (e-mail: \{k.s.yildirim, p.pawelczak, a.y.majid\}@tudelft.nl).
	\IEEEcompsocthanksitem Henko Aantjes is with the Brighsight BV, Brassersplein 2, 2612\,CT, Delft, The Netherlands (e-mail: aantjes@brightsight.com)
	\IEEEcompsocthanksitem Kas{\i}m Sinan Y{\i}ld{\i}r{\i}m is also affilated with the Computer Engineering Department of Ege University, Izmir, Turkey.
	}
%	\thanks{Manuscript received April 19, 2005; revised August 26, 2015.}
}

\maketitle

\IEEEdisplaynontitleabstractindextext

\IEEEpeerreviewmaketitle


\appendices
\section{Analysis of Algorithm 1}

We provide the synchronization conditions of Algorithm 1 and show that the synchronization will be established eventually. We need to mention that we follow similar calculation steps to the ones presented in the studies~\cite{pi2015,Yildirim:Gradient:2016}. However, since we use a different clock drift model in this article, i.e. Brownian motion rather than uniform distribution, the following  mathematical analysis is moderately different and more realistic as compared to those in the previous studies. 

Since $f_w$ and $f_r$ are time varying, by following 
\cite{Zhong:2011:On-demand,Huang:2013:PSR}, we model the 
\emph{evolution} 
of $f_w^r$ as 
integrated white noise, i.e. \emph{Brownian motion}, given as
\begin{eqnarray}
f_{w}^{r}(t_{0}+\delta) & \triangleq & 
f_{w}^{r}(t_{0})+\int_{t_{0}}^{t_{0}+\delta}\eta(t)dt.\label{eq:drift}
\end{eqnarray}
Here, $\delta$ denotes the amount of time after $t_0$ and $\eta(t)$ denotes 
the instantaneous change on the relative clock frequency, namely 
\emph{relative drift rate} at time $t$. We assume that 
$\eta(t)\sim\mathcal{N}(0,\sigma_{\eta}^{2})$ and for all 
time instants $t'$ and $t''$, $\eta(t')$ and $\eta(t'')$ are uncorrelated 
random variables. In order to ease the calculations in the following sections, 
we will consider the \emph{average relative clock frequency} in an interval of 
interest $[t_0,t_0+\delta]$ rather than instantaneous relative frequency 
in~\eqref{eq:drift}. We denote this value by $\bar{f}_{r}^{w}(t_0,t_0+\delta)$ 
and define it as 
\begin{eqnarray}
\bar{f}_{w}^{r}(t_0,t_0\!+\!\delta) &\!\!\!\!\! \triangleq\!\!\!\!\! & 
\frac{1}{\delta}\!\!\int_{t_0}^{t_0+\delta}\!\!\!\!\!\!\!\!\!\!\!\! 
f_w^r(t)dt \triangleq
f_{w}^{r}(t_{0})\!+\!\frac{1}{\delta}\!\!\int_{0}^{\delta}\!\!\int_{t_{0}}^{t_{0}+t}\!\!\!\!\!\!\!\!\!\!\!\eta(u)dudt.
\label{eq:drift_relation}
\end{eqnarray}


\subsection{Proof of Convergence}
Consider (5) and without loss of 
generality we assume that 
$C_{w}(t_{1})-C_{w}(t_{0})=\tau(1+\bar{f}_{w}^{r}(t_0,t_1))+\varepsilon$
where $\tau$ denotes the event period and $\varepsilon$ accounts for the 
uncertainty of the event period (as 
indicated in Section 6.1). For ease of 
calculations we assume that $\varepsilon\sim\mathcal{N}(0,\sigma^2_{\tau})$. 
Define \emph{relative frequency estimation error} as
\begin{eqnarray}
\tilde{f}_{w}^{r}(t_0) & \triangleq & 
f_{w}^{r}(t_0)-\hat{f}_{w}^{r}(t_0).\label{eq:relative_drift_estimation_error}
\end{eqnarray}
Using (5), \eqref{eq:drift_relation} and 
\eqref{eq:relative_drift_estimation_error} we get:
\begin{eqnarray}
\gamma(t_1) & = & 
\tau(\bar{f}_{w}^{r}(t_0,t_1)-\hat{f}_{w}^{r}(t_0))+ \varepsilon \nonumber \\
&=&  
\tau\tilde{f}_{w}^{r}(t_0)+\frac{\tau}{t_1-t_0}\int_{0}^{t_1-t_0}\!\!\!\int_{t_{0}}^{t_0+t}\!\!\!\!\!\!\!\!\!\!\!\eta(u)dudt
+ \varepsilon.
\end{eqnarray}
For ease of representation, let us introduce the following 
notation:
\begin{eqnarray}
\Phi(h) &\triangleq& \int_{t_h}^{t_{h+1}}\eta(t)dt, \\
\Omega(h) &\triangleq& 
\frac{\tau}{t_{h+1}-t_h}\int_{0}^{t_{h+1}-t_h}\int_{t_{h}}^{t_h+t}\eta(u)dudt,
\\
\Delta(h) &\triangleq& \tilde{f}_{w}^{r}(t_{h}). \label{eq:delta}
\end{eqnarray}
Using this notation, we denote $\gamma(t_{h+1})$ by $\Theta(h+1)$ as
\begin{eqnarray}
\Theta(h+1) &\triangleq& \gamma(t_{h+1})=\tau\Delta(h) + \Omega(h)+\varepsilon. 
\label{eq:theta}
\end{eqnarray}
Moreover, since $\Delta(h)$, $\Omega(h)$ and $\varepsilon$ are 
independent random variables, we can write $\Delta(h+1)$ as
\begin{eqnarray}
\Delta(h+1) &\overset{\eqref{eq:delta}}{=}& 
\Delta(h)+\tilde{f}_{w}^{r}(t_{h+1})-\tilde{f}_{w}^{r}(t_{h}) \nonumber \\
&\overset{\eqref{eq:relative_drift_estimation_error}}{=}& \Delta(h)
\!+\!f_w^r(\!t_{h+1}\!)\!-\!\!f_w^r(\!t_{h}\!) \!-\!\!\hat{f}_w^r(\!t_{h+1}\!)\!+\!\!\hat{f}_w^r(\!t_{h}\!)
\nonumber\\
&\overset{(6),\eqref{eq:drift}}{=}& 
\Delta(h)+\int_{t_h}^{t_{h+1}}\eta(t)dt-\beta\Theta(h+1) \nonumber \\
&\overset{\eqref{eq:theta}}{=}& 
(1-\beta\tau)\Delta(h)-\beta(\Omega(h)+\varepsilon) +\Phi(h). 
\label{eq:delta_recursive}
\end{eqnarray} 
Considering \eqref{eq:theta} and \eqref{eq:delta_recursive}, we can write the 
system evolution as
\begin{eqnarray}
\underset{X(h+1)}{
	\underbrace{
		\left[\!\!\!\begin{array}{c}
		\Theta(h\!+\!1)\\
		\Delta(h\!+\!1)
		\end{array}\!\!\!\right]
	}
} &\!\!\!\!\!\!\! =\!\!\!\!\!\!\! & 
\underset{A}{
	\underbrace{
		\left[\!\!\!\begin{array}{cc}
		0 & \tau\\
		0 & 1\!-\!\beta\tau
		\end{array}\!\!\!\right]
	}
}
\underset{X(h)}{
	\underbrace{
		\left[\!\!\!\begin{array}{c}
		\Theta(\!h\!)\\
		\Delta(\!h\!)
		\end{array}\!\!\!\right]}
}
\!\!+\!\!
\underset{Y(h)}{
	\underbrace{
		\left[\!\!\!\begin{array}{c}
		\Omega(h)+\varepsilon\\
		\Phi(\!h\!)\!-\!\beta(\Omega(\!h\!)\!+\!\varepsilon)
		\end{array}\!\!\!\right]}
}.\label{eq:state_space-pi}
\end{eqnarray}
Taking the expectation of both sides of \eqref{eq:state_space-pi} yields 
$\expected{X(h+1)} = A\expected{X(h)}$ since 
the means of the random variables in the matrix $Y(h)$ are all zero. According 
to stability of linear systems \cite{book-luenberger}, \emph{asymptotic 
	convergence} will be reached if and only if the magnitudes of the 
eigenvalues of the matrix $A$ are strictly smaller than one. The eigenvalues 
$\lambda_1$ and $\lambda_2$ of the matrix $A$ can be obtained 
by solving
\begin{eqnarray}
\left|\begin{array}{cc}
-\lambda & \tau\\
0 & 1-\beta\tau-\lambda
\end{array}\right|=0 &\implies& 
\begin{array}{c}
\lambda_1 = 0,\\
\lambda_2 = 1-\beta\tau.  
\end{array}\label{eq:eigens}
\end{eqnarray}
Therefore, the synchronization will be  established if 
and only if $|\lambda_1,\lambda_2|<1$, and in turn
\begin{align}
0 & <\beta<\frac{2}{\tau}\label{eq:step_size_bounds-pi}
\end{align}
should hold. The 
rate of convergence is governed by 
the largest eigenvalue $\lambda_2$, which means that bigger values of $\beta$ 
will lead to faster convergence. 
Consequently, selecting $\beta$ within the bound above will lead the system  
converge to the state  
\begin{eqnarray}
\underset{h\rightarrow\infty}{\lim}\expected{\Theta(h+1)} &=&
\expected{\Theta(h)}, \label{eq:asymp_theta} \\
\underset{h\rightarrow\infty}{\lim}\expected{\Delta(h+1)}&=&
\expected{\Delta(h)} \label{eq:asymp_delta}.
\end{eqnarray}
Hence, due to asymptotic convergence
\begin{eqnarray}
\expected{\Delta(\infty)} \overset{\eqref{eq:delta_recursive}}{=} 
\left(1-\beta 
\tau\right)\expected{\Delta(\infty)} &\!\!\!\!\!\!\!\implies\!\!\!\!\!\!\!& \expected{\Delta(\infty)}\!=\!0, 
\label{eq:delta_convergence}\\
\expected{\Theta(\infty)}  \overset{\eqref{eq:theta}}{=} 
\tau\expected{\Delta(\infty)}&\!\!\!\!\!\!\!\implies\!\!\!\!\!\!\!&\expected{\Theta(\infty)}\!=\!0. 
\label{eq:expected_error}
\end{eqnarray}
Consequently, the prediction error $\Theta(h)=\gamma(t_h)$ 
eventually converges to zero as $h$ goes to infinity. This means the WISP tag 
will estimate the occurrence times of the successive
events with zero error in expectation. 


\subsection{Steady-State (Asymptotic) Synchronization Error}
\label{sec:app_b}
An analytical expression for the \emph{asyptotic 
	synchronization error} can be obtained by calculating 
$\underset{h\rightarrow \infty}{\lim}\mathrm{Var}(\Theta(h))$. This 
calculation reduces to find 
$\mathrm{Var}(\Theta(\infty))=\expected{\Theta(\infty)^2}$, since it holds 
from 
\eqref{eq:expected_error} that $\expected{\Theta(\infty)}=0$. Using 
\eqref{eq:theta}, we 
get
\begin{eqnarray}
\expected{\Theta(\infty)^2} &\!\!\!\!\overset{\eqref{eq:theta}}{=}\!\!\!\!& 
\tau^2\expected{\Delta(\infty)^2}  
+\expected{(\Omega(h)+\varepsilon)^2},\label{eq:expected_theta_infinity}
\end{eqnarray}
since $\expected{\Delta(\infty)}=0$ from \eqref{eq:delta_convergence} and 
$\Delta(h)$, $\Omega(h)$ and $\varepsilon$ are independent. 
In order to calculate \eqref{eq:expected_theta_infinity}, let us derive first 
$\expected{\Delta(\infty)^2}$ as
\begin{eqnarray}
\!\!\!\!\!\expected{\!\Delta(\!\infty\!)^2\!} 
&\!\!\!\!\!\!\!\overset{\eqref{eq:delta_recursive},\eqref{eq:asymp_delta}}{=}\!\!\!\!\!\!& 
(\!1\!\!-\!\!\beta\tau\!)^2\!	
\expected{\!\Delta(\!\infty\!)^2}  
\!\!+\!\!\expected{\!(\!\Phi(\!h\!)\!\!-\!\!\beta\!(\!\Omega(\!h\!)\!+\!\varepsilon\!)^2}\!. 
\label{eq:expected_delta_infinity}
\end{eqnarray}
In order to calculate \eqref{eq:expected_delta_infinity}, the first step is to 
obtain the following expectation:
\begin{eqnarray}
\expected{(\Phi(h)\!-\!\beta(\Omega(h)\!+\!\varepsilon)^2} 
&\!\!\!\!=\!\!\!\!&
\expected{(\Phi(h)^2}
\!-\!2\beta\expected{\Phi(h)\Omega(h)} 
\! \nonumber \\ 
& & +\!\beta^2\expected{\Omega(h)^2}\!+\!\beta^2\expected{\varepsilon^2}. 
\label{eq:stochastic_expectation}
\end{eqnarray}
The equality \eqref{eq:stochastic_expectation} holds due to the fact that 
$\expected{\Phi(h)\varepsilon}=\expected{\Phi(h)}\expected{\varepsilon}=0$ and 
$\expected{\varepsilon\Omega(h)}=\expected{\varepsilon}\expected{\Omega(h)}=0$ 
(since 
$\expected{\varepsilon}=\expected{\Phi(h)}=\expected{\Omega(h)}=0$ and 
$\varepsilon$ is independent from $\Phi(h)$ and $\Omega(h)$). The
following \emph{stochastic integrals} are required for 
\eqref{eq:stochastic_expectation}:
\begin{eqnarray}
\expected{\Phi(h)^2}\!\!\!\!\!\!\! 
& = & \!\!\!\!\!\!
\int_{0}^{t_{h+1}-t_h}\!\!\!\!\!\!\!\!\!\!\!\!\!\!\!\!\!\!\sigma_{\eta}^{2}dt =  \sigma_{\eta}^{2}(t_{h+1}-t_h),\\
\expected{\Omega(h)^2}\!\!\!\!\!\!\! 
& = & \!\!\!\!\!\!
\frac{\tau^2}{(t_{h+1}\!\!-\!\!t_h)^2}
\sigma_{\eta}^{2}\!\!\!\int_{0}^{t_{h+1}-t_h}\!\!\!\!\!\!\!\!\!\!\!\!\!\!\!\!\!\!\!t^2dt\! = \!
\sigma_{\eta}^{2}\!\frac{\tau^2(\!t_{h+1}\!\!-\!\!t_h\!)}{3},\label{eq:stochastic_integral}\\
\expected{\Phi(h)\Omega(h)}\!\!\!\!\!\!\! 
& = & \!\!\!\!\!\!
\frac{\tau}{t_{h+1}-t_h}\sigma_{\eta}^{2}\int_{0}^{t_{h+1}-t_h}\!\!\!\!\!\!\!\!\!\!\!\!\!\!\!\!\!\!tdt = 
\sigma_{\eta}^{2}\frac{\tau(\!t_{h+1}\!\!-\!\!t_h\!)}{2}, \\
\expected{\varepsilon^2}\!\!\!\!\!\!\! 
& = & \!\!\!\!\!\! 
\sigma_{\tau}^{2}.
\end{eqnarray}	
The derivation of the expectations and stochastic integrals above relies on the 
assumptions we mentioned 
at the beginning of this Appendix: (i) $\eta(t')$ and $\eta(t'')$ are 
uncorrelated for all $t'\neq t''$, therefore $\expected{\eta(t')\eta(t'')}=0$; 
(ii) since $\expected{\eta(t')}=0$, therefore 
$\expected{(\eta(t'))^2}=\mathrm{Var}(\eta(t'))=\sigma_{\eta}^2$. Putting these 
expected 
values into \eqref{eq:stochastic_expectation}, then in 
turn into \eqref{eq:expected_delta_infinity} lead:
\begin{eqnarray}
\expected{\Delta(\infty)^2}\!\!\!\!\!\!
& = &\!\!\!\!\!\!
\frac{\sigma^2_{\eta}(t_{h+1}-t_h)\left(1-\beta\tau+\frac{\beta^2\tau^2}{3}\right)\!\!+\!\!\beta^2\sigma^2_{\tau}}{(2\beta\tau-\beta^2\tau^2)}.
\label{eq:delta_infinity}
\end{eqnarray}
Finally, putting \eqref{eq:delta_infinity} into 
\eqref{eq:expected_theta_infinity} 
leads to
\begin{eqnarray}
\expected{\Theta(\infty)^2}\!\!\!\!\!\!\! & = & \!\!\!\!\!\!\!
\frac{\sigma^2_{\eta}(t_{h+1}-t_h)\left(\tau^2
	-\beta\tau^3-\frac{\beta^2\tau^4}{3}\right)
	+\beta^2\sigma^2_{\tau}\tau^2}{(2\beta\tau-\beta^2\tau^2)} \nonumber \\ 
\!\!\!\!\!\!\! &  & \!\!\!\!\!\!\!+ 
\sigma^2_{\eta}\frac{\tau^2(t_{h+1}-t_h)}{3}+\sigma^2_{\tau}.
\end{eqnarray}
Without loss of generality, assume that $t_{h+1}-t_h\approxeq\tau$. Therefore, 
the asymptotic estimation error of Algorithm 1 is given by
\begin{eqnarray}
Var\left(\underset{h\rightarrow\infty}{\gamma(t_h)}\right)\!\!\!\!\!\!\!& = &\!\!\!\!\!\! 
\expected{\Theta(\infty)^2}\!\! = \!\!
\frac{\sigma^2_{\eta}\left(\tau^2-\beta\tau^3
	+\frac{\beta^2\tau^4}{3}\right)
	+\beta^2\sigma^2_{\tau}\tau}{(2\beta-\beta^2\tau)} \nonumber \\
\!\!\!\!\!\!\!& = &\!\!\!\!\!\!  +\frac{\sigma^2_{\eta}\tau^3}{3}
+  \sigma^2_{\tau}
\end{eqnarray}
which results in (7).


\bibliographystyle{IEEEtran}
\bibliography{IEEEabrv,references}

\end{document}


