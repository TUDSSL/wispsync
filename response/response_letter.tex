\documentclass[10pt]{article}
\usepackage{amsmath,amsfonts,amssymb,graphicx,color,psfrag}
\usepackage[utf8]{inputenc}

\usepackage{vmargin}
\setmarginsrb{2.25cm}{2.25cm}{2.25cm}{2.25cm}
         {12pt}{20pt}{12pt}{36pt}

\usepackage{soul}
\usepackage{hyperref}

\newcommand{\referee}[1]{
	{\item \color{OliveGreen} \emph{{#1}}}
	\label{R\therefereeCounter:\arabic{enumi}}
}

\definecolor{OliveGreen}{rgb}{0.1,0.2,0.5}

\newcommand{\response}[1]{{\color{blue} #1}}

\newenvironment{editor}{%
\textbf{\large Comments by the Editor}
\begin{enumerate}%
\renewcommand{\labelenumi}{\textbf{[E:\,\arabic{enumi}]}} %
}{\pagebreak\end{enumerate}}

\newcounter{refereeCounter}

\newenvironment{responses}{%
\refstepcounter{refereeCounter}%
\textbf{\large Comments by Referee \therefereeCounter}
\begin{enumerate}%
\renewcommand{\labelenumi}{\textbf{[R\therefereeCounter :\,\arabic{enumi}]}} %
}{\end{enumerate}}

\setlength{\parindent}{0pt}

\begin{document}

\pagestyle{myheadings}
\thispagestyle{empty}

\markright{\footnotesize TMC-2017-11-0695: {\sl On the Synchronization of Computational RFIDs}}

\headsep 0.5cm

\bigskip\bigskip

%\noindent{Marwan Krunz, EIC, and Kevin Almeroth, Associate EIC \\
%	IEEE Transactions on Mobile Computing
%}

\bigskip\bigskip

\begin{flushright}
Dr. Przemys{\l}aw Pawe{\l}czak \\
Embedded Software Group \\ 
Delft University of Technology \\ 
Mekelweg 4, 2628 CD, Delft, The Netherlands \\
E-mail: p.pawelczak@tudelft.nl\\
Phone: +31 15 278 7491\\
\end{flushright}

\vspace*{2cm}

\today
\medskip


\textbf{TMC-2017-11-0695:} {\sl On the Synchronization of Computational RFIDs}

\bigskip

Dear Editor,\\

We would like to thank for handling our paper and providing a rapid review results. We are also grateful to the Reviewers for their substantial effort spent for evaluating our work. \\

With this letter we provide our response to all comments of the Reviewers and the revised version of our paper. We look forward to hearing from you at the earliest and we would be happy to provide further clarifications and revisions if necessary. \\

With best regards, \\

in the name of all Authors---Dr. Przemys{\l}aw Pawe{\l}czak

\pagebreak

\begin{editor}
\referee{The authors need to revise the paper accordingly and prepare for the second review. The reviewers mainly concerns in the following aspects.
	\begin{enumerate}
		\item Clearly explain the motivation, describe clearly about the problem definition, and describe what new challenges address in this paper;
		\item Address the technical details mentioned by Reviewers 1 and 2;
		\item More results to present mentioned by Reviewer 3.
\end{enumerate}
}

\response{
	Thank you once more for coordinating the review process of our paper. We hope that we have revised our article in accordance with the responses of the reviewers. We therefore believe we have addressed all three points that the Editor raised as accurately as possible. We request the Editor to refer to our respective responses below. \textcolor{red}{Add direct references where each of these three points has been addressed.}
}

\end{editor}

\begin{responses}
 
\referee{In the Introduction part, the authors should clarify what type of synchronization they care. There are two types of synchronization of RFID communication: syn. between reader and tag; syn. among multiple tags. The COTS RFID devices exploit the EPC C1G2 standard, which uses framed slotted Aloha protocol to query tags. We found that the tag reading performance is very well. Hence, I think the synchronization among tags is not a big problem that hinder RFID communication. Hence, this study is not well motivated a least currently.}

\response{
We thank the reviewer for pointing this issue that led us to emphasize our motivation in the revised article. As indicated by the reviewer, framed slotted Aloha protocol indeed performs a form of synchronization: by adjusting offset and introducing guard times, tags are ensured to communicate at the allocated frame boundaries. However, 

\begin{itemize}
	\item this method \emph{does not output a synchronized clock value} that can be accessed by the tags to timestamp (and make inference about) the collected information, e.g. from the sensors attached to the CRFID;
	\item in consequence, \emph{applications requiring time notion cannot be enabled}, since a clock is not maintained at the CRFID tags.
\end{itemize}

Unfortunately due to frequent power failures and marginal energy budgets of CRFIDs, providing explicit clock synchronization (that represents \emph{synchronized notion of time}) and maintaining this clock value is challenging. We have emphasized these points in our revised article in Section 1.1 and Section 2.1.3. We hope that with these answers we have motivated the need for our study sufficiently. We kindly refer the reviewer also to our responses \hyperref[R1:2]{[R1:2]} and \hyperref[R1:4]{[R1:4]}. \textcolor{red}{Add clarification on the difference between clock synchronization with Aloha and our study in Section 1.1. and 2.1.3}
}

\referee{Also, there are another view to classify the synchronization problems: slot level synchronization (time frame is divided into slots) or bit level synchronization. Slot level synchronization has been studied for decades and is well addressed. The bit-level synchronization is not well investigated yet and attracts much attention from academic communities, e.g., the following three references. The authors should discuss and clarify whether they study the bit-level synchronization mentioned in these references.
}

\response{
We thank the reviewer for this comments and for pointing us to these relevant papers, which we cite in Section 2.1.3 of our revised work. Thanks to the comment of the reviewer we have noticed that in the previous version of our submission we did not highlighted the main difference between our work and the works pertaining to the \emph{bit-level synchronization}.

To clarify this difference: as indicated earlier by the reviewer, it is possible to exploit modulation properties of signals from the reader to the tags (such as bit timing in FM0 modulation) in order to get the timing measurements of the regular patterns in the signal, i.e. bit-level synchronization. However, the objective of our work has the following fundamental differences as compared to these studies:

\begin{itemize}
	\item \textbf{Portability:} Bit-level synchronization relies on high-precision clock to obtain microsecond-precision timing measurements. Despite, in our work, we used the external 32\,kHz clock that can run in low-power standby mode and the \emph{higher-level} commands of the EPC Gen 2 standard, rather than exploiting its low-level signal properties. Therefore, our approach is independent of how EPC Gen2 is implemented---providing a level of portability across different platforms. However, our synchronization algorithms can also be integrated into the bit-level synchronization;
	
	\item \textbf{Explicit Clock:} Bit-level synchronization does not output an explicit clock value and it does not compensate for the clock drift. Rather, it uses offset compensation only and \emph{guard time} to reduce the effect of the clock drift. On the contrary, we provided time synchronization algorithms that compensate for the clock drift and provide an \emph{explicit} common time notion to CRFID applications;
	
	\item \textbf{Energy Budget:} Due to marginal energy budgets, we proposed synchronization algorithms that require minimum number of computation steps and memory space;
	
	\item \textbf{Power Failures:} Providing synchronization service to the application layer requires to cope with power failures, which has not been investigated so far. To address this issue, we provided synchronization algorithms that can recover easily after power failures.
\end{itemize}

We would like to emphasize that none of these issues have been addressed and studied so far within the context of CRFIDs. The above discussion has been briefly added to Section 2.1.3. \textcolor{red}{Add brief extension on clock synchronization to Section 2.1.3}
}

\referee{What if WISP tags are simultaneously covered by multiple readers. In this case, the readers are not synchronized themselves and may have different time-stamps. Then, how does the proposed method work?}

\response{
As we indicated in the introduction of the article (Section 1) there are already existing synchronization techniques that provide synchronization among RFID readers. Therefore, if there are multiple readers that cover multiple WISP tags, first the readers should be synchronize with each other and then the tags will be synchronized to the RFID readers. This approach will cause any danger to the protocol we designed. We mentioned this explicitly in Section 4 of our revised paper. \textcolor{red}{Add clarification about single reader synchronization}
}

\referee{The novelty of this paper is relatively limited because the proposed event-based reader-tag synchronization method is originated from previous literature. Moreover, the main part of the theoretical analytics is also from previous work [24,31]. Although the authors claimed that they use a different clock drift model in this article, i.e. Brownian motion rather than uniform distribution, I still this significantly reduces the theoretical depth of this paper. The authors should show what NEW challenges beyond previous work are addressed by this paper. }

\response{
As we indicated in our previous responses (see \hyperref[R1:1]{[R1:1]} and \hyperref[R1:2]{[R1:2]}), our article has the following novelties over the existing work in the literature:
\begin{itemize}
	\item Bit-level synchronization does not compensate for clock drift and it does not provide explicit time notion to the tags. In order to output a synchronized clock value, a time synchronization algorithm is required.
	\item Since the energy budget of the tags are limited, synchronization approaches that require marginal number of computation and memory. This is challenging---the main motivation for applying a PI-controller based synchronization for the first time in this domain.
	\item Maintaining an explicit synchronized clock despite power failures has never been explored before in the field of CRFIDs. We also provided how our synchronization techniques overcome frequent power failures and provide a stable synchronized clock value.
	\item  We also provided an in-depth theoretical analysis of the proposed synchronization techniques. As indicate by the reviewer, the main part of the analysis is from previous work. However, the previous analysis that assumes uniform distribution for the clock drifts was not realistic---the reason we changed the analysis to use Brownian motion. This modification did not change the proof structure but it significantly changed the individual steps of the proof---that led to different stead-state error expressions.
\end{itemize}
}

\end{responses}

\begin{responses}
	
\referee{In this paper, the authors study the time synchronization problem in low-power backscatter systems, i.e., WISP. They point out the unique challenges in designing time synchronization algorithms for these energy-harvesting tags, and propose two synchronization protocols in response to these challenges. The proposed protocol is compatible with the default reader-to-tag communication protocol and consumes an ignorable amount of energy. The authors evaluate their proposed protocols through real-world experiments. \\
\\
+: an interesting problem which has not been explored before. \\
+: the proposed protocols are simple yet look effective. \\
+: evaluating the proposed protocols with real-world experiments; \\
\\		
-: the motivation behind is not well explained; \\
-: not sure the proposed protocol will work in large-scale deployment. \\
\\
My major concern about this paper is its motivation: I do agree with the authors that the time synchronization is important to the multi-hop backscatter networks, but I don’t think it is necessary for the single-hop network like the WISP networks. The ALOHA-type protocol has already synchronized the tags in nature. 
}
	
\response{ 
 The concern of this reviewer has been shared by also Reviewer 1. In our revised article, we provided a better motivation and highlighted the novelty of our work better.  \\
 In the previous responses---please also see \hyperref[R1:1]{[R1:1]} and \hyperref[R1:2]{[R1:2]}---we mentioned that ALOHA-type protocol does not maintain an explicit clock value and does not implement a real synchronization algorithm. On the contrary, we provided novel synchronization mechanisms that enables tags to access synchronized time---this also enables sensor network-like data collection applications where tags collect and timestamp sensor readings that can be queried by the readers later. However, maintaining an explicit clock and ensuring synchronization is not trivial due to the marginal energy budgets of the tags and frequent power failures. These are the main challenges of our work and its novelty over existing studies. 
}
	
\referee{I’m not sure synchronizing tags through the command initiated by the 
excitation generator (e.g., RFID reader) is feasible in real-world applications. This is because these energy-harvesting sensors can be charged within a very short range. Hence if the sensors are deployed sparsely, multiple excitation generators are needed, and it is thus important to synchronize these excitation generators as well. However, the authors did not consider this point.
}
	
\response{
We thank the reviewer for raising this issue. We already answered this concern in the previous responses---please also see \hyperref[R1:3]{[R1:3]}. We would like to mention that there are already prior work that provides synchronization among the readers. Therefore, we can also employ these strategies in order to create a common time notion among the readers. In order to provide an explicit synchronized clock value to the WISP tags, an explicit reader-tag synchronization algorithm is required---which is the focus of our work.
}
	
\referee{Some questions about the technical parts: \\
1): the authors claim that the fluctuating input voltage prevents short-term stability of the clock hardware and introduces significant clock drift in WISP. However, as far as I know, WISP uses a regulator to supply a constant 1.8\,V DC to power the WISP.}
	
\response{
	Fill
}
	
\referee{2): section 5.1, the authors observe a 1.89\,ms transmission delay. What does this delay account for? the in-air time should not be so long.
}
	
\response{
In our article, we defined the \emph{transmission delay} as the time that passes between the start of the broadcast command from the reader and the receipt by the receiver node--- which is the major error source and composed of non-deterministic components. Since we used high-level properties of EPC Gen2 protocol, we could not use low-level timestamping mechanisms and our synchronization approaches suffer from big transmission delays.  The reason is that we do not know when the reader starts transmission after broadcast command is requested from the PC that controls the reader. During this process, the reader-side does not implement any timestamping mechanism and the delays are mainly dependent on the reader-side implementation of the EPC protocol.   
}
	
\end{responses}

\begin{responses}

\referee{It is better to simplify the descriptions of the challenges and existing works, especially some contents that are not directly related to the synchronization.}
	
\response{
	To address the comment of the reviewer we have reduced the length of the description in Section 2 of the paper. \textcolor{red}{Compress text in Section 2}
}

\referee{It is better to report some results or discuss the performance of the proposed approach in the multi-hop networks, since the authors highlights the advantages of  multi-hop networks several times in the introduction.}
	
\response{
	Fill \textcolor{red}{Matlab simulations for the multi-hop scenario.}
}
	
\referee{The authors may consider to discuss a specific scenario that can use the proposed approach.}
	
\response{
	Fill \textcolor{red}{Discuss a scenario for our method}
}

\end{responses}

\end{document}